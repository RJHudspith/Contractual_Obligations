\section{Utilities and general Linear Algebra}

In this section we discuss some of the more nitty-gritty aspects
of the code and general linear algebra and contractions that the
code performs.

The code also has SSE (2 and 3) variants of the operations in an
attempt to accelerate several aspects of the code. It is partially
successful at doing so.

\subsection{(Color) matrix operations}

We define a colo matrix as an $NC\times NC$ complex matrix.

As our propagator is stored as dirac,dirac,color,color we can
perform color-color operations with ok cache coherence. This
became clear when we needed to take color products with gauge
fields for the non-local currents.

Again from Jamie's previous experience it is beneficial to
have the following matrix operations hard-coded (a,b and c
are color matrices),
\begin{equation}
\begin{gathered}
a=b, \quad a+=b, \quad \text{Tr}\left[ a \right], \quad \text{Tr}\left[ a.b \right], 
\quad a=B*c, \quad a=b^{\dagger}, \\
a=b.c, \quad a=b.c^{\dagger}, \quad a=b^{\dagger}.c, \quad a=b^{\dagger}.c^{\dagger}.
\end{gathered}
\end{equation}
Where $a=B*c$ is the product of a complex constant ``B'' multiplied
by the color matrix c.

In several cases these functions have a lower operation count than
their brute force counterparts.

\subsection{Spinor operations}

We define a spinor as an $NS \times NS$ matrix of color matrices,
with sink index running fastest and source index slowest.

As of writing we support the following spinor operations (S1 
and S2 amd S3 are spinors, U is a color matrix)
\begin{equation}
\begin{gathered}
S1=S2, \quad S1=-S2, \quad S1 = U.S2, \quad S1 = U^{\dagger}.S2, \\
\quad S1 = S2.U^{\dagger}, \quad S1=U.S2^{\dagger}, \\
\quad S1 = S2.U , \quad S1 = S2^{\dagger}.U, \quad S1=S2.U^{\dagger},\\
S1=\sum_{x}^{V_{ND-1}}S2[x], \quad S1 = S2.S3, 
\quad S1[x] = 0 \; (x\in V_{ND-1}), \quad S1=0.
\end{gathered}
\end{equation}
Where the dagger operation is only over the color matrices,
although I am no longer sure what the point of most of these
operations was.

Also, $V_{ND-1}$ is our notation for the spatial-only subvolume.
i.e. the sum is over a time-slice.

\subsection{Contractions}

Our code has some fairly clever gamma-spinor type stuff going
on. Our code has the following operations encoded (again
S1 and S2 are spinors, B is a complex number),
\begin{equation}
\begin{gathered}
S1 = S2^{\dagger}, \quad B=\text{Tr}\left[ S1.S2 \right], 
\quad S1 = \left( \gamma_5.S2.\gamma_5 \right)^{\dagger},\quad \\
S1 = \gamma_i.S2, \quad S1 = S2.\gamma_i,
\quad S1 = \gamma_i.S2.\gamma_j, \\
\quad B = \text{Tr}\left[ \gamma_j.\left( \gamma_5.S2.\gamma_5\right)^{\dagger}.\gamma_i.S1\right], \\
\quad B = \text{Tr}\left[ \gamma_j.S2.\gamma_i.S1\right], \\
\end{gathered}
\end{equation}

Again, the motive for having both left multiply, right multiply
and left-right multiply by gamma matrices is that the operation
count is reduced for the left-right multiply.

The final two spinor operations are meson traces for contracting
($\gamma_5$-Hermitian) spinors correctly to form a meson.

Unlike in the spinor operations subsection, the $\dagger$ is
now acting over all spin-color indices as is the trace.

\subsection{Baryon operations}

\commentj{Anthony, could you have a crack at this when you have time?}

\subsection{Baryon contractions}

\commentj{Anthony, could you have a crack at this when you have time?}

\subsection{Basis rotations}

If we are contracting a propagator in the chiral propagator
with a non-relativistic propagator we rotate the chiral
propagator into the non-relativistic basis before contracting
the two propagators.

\commentj{Definition of the rotation matrix should go here}


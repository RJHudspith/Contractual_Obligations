\section{Tests}

This section deals with the stuff in the tests directory, namely
the unit tests and meson extractor.

\subsection{Unit tests}

We perform unit tests on the basic linear algebra of our library
to ensure the basic stuff is correct. Higher level functions
such as the meson and baryon correlators are up to you.

The unit testing framework is built from minunit %\href{}
which is the simplest possible framework (it is literally
two macros). We decided on this because we didn't want the user
to have to install any extra code just to check ours.

The tests either all pass or break at a specific point, returning 
the error message that was produced. If you change any of the
contractions codes in UTILS we strongly recommend that the
unit tests are run to make sure everything is kosher.

If you add some extra function to the files already covered by
the unit tests please add a unit test for it.

\subsection{Meson extractor}

The meson extractor is a simple code to look directly at the
correlator files our code outputs. The output files the code
produces are binary data with a spatial momentum list first and
then the full correlator data afterwards.

If I wanted to extract the Pion with 1,1,0 momentum projection
from the file ``mesons.bin'' I would call,
\begin{verbatim}
./MESONS mesons.bin 5,5,1,1,0
\end{verbatim}
With the first two indices being the source and sink gamma indices
which for our gamma convention is $\gamma_5 \gamma_5$. The momenta
follows our conventions for geometry too, with px running fastest
and in this example pz running slowest.

\subsection{VPF extractor}

We also provide a small vacuum polarisation function reader for our
data files. The momentum ordering is the same $(p_x,p_y,p_z,p_t..)$
only we have taken the full Fourier transform over the whole volume
for this data.

The extractor requires 3 arguments (for example opening a file 
``cl.CVLV.trans.bin'' with geometry $16x16x16x16$),
\begin{verbatim}
./VPFREADER cl.CVLV.trans.bin 16,16,16,16
\end{verbatim}
Will check the checksums and output all possible momenta along
with $(a\hat{q})^2$ and $\Pi^{(1)}((a\hat{q})^2)$. An example would
be,
\begin{verbatim}
....................
[MOMS] 5325 :: ( -5  -5  7  7 ) 13.226252 -0.003262 
[MOMS] 5326 :: ( 5  -5  7  7 ) 13.226252 -0.003262 
[MOMS] 5327 :: ( -5  5  7  7 ) 13.226252 -0.003262 
[MOMS] 5328 :: ( 5  5  7  7 ) 13.226252 -0.003262
....................
\end{verbatim}
The first integer is the momentum index number, the 
Fourier modes, then $(a\hat{q})^2$ and $\Pi^{(1)}((a\hat{q})^2)$.

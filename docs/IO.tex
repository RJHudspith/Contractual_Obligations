\section{IO}

Our code sequentially reads the propagator timeslice-by-timeslice
as a memory saving technique.

\subsection{Our propagator format}

The propagator file our code expects conforms to our lattice geometry
(i.e. $x,y,z,t$ where x runs fastest and t slowest). It is expected
to be in $\text{dirac}_1\text{dirac}_2\text{color}_1\text{color}_2$
order. i.e. the full $NS*NS*NC*NC$ propagator matrix where 1 is the
source index and 2 is the sink. It also has the following 
information in the header,

\begin{verbatim}
<start_header>
Lattice:  16 16 16 16
Plaq: 1.000000
OutProp:  6.000000 0.125000 1.000000
SrcPos:   0 0 0 0
Basis: Chiral/Nrel_fwd/Nrel_bwd
Source:  Point
Smearing: None/<enter favourite>
Gaugesmearing: None/<enter favourite>
Endian: Little
Precision: Double
<end_header>
\end{verbatim}

First of all, the \verb|tag: | the pattern is tag colon space stuff.
\verb|Lattice| is $(x,y,z,t,..)$ i.e. if we are compiled for $ND=5$ it will
expect 5 arguments here and fail if it doesn't get them.

\verb|Plaq| is the plaquette of the gauge field this propagator was measured
on.

\verb|OutProp| is some general information about the prop such as $\kappa$.

\verb|SrcPos| is the source position

\verb|Basis| is the gamma basis needed for this propagator, i.e. is it a
non-relativistic propagator or a chiral propagator.

\verb|Source| can be \verb|Point,Wall,Z2WALL|

\verb|Smearing| Fermion smearing information

\verb|Gaugesmearing| Did we perform gauge field smearing

\verb|Endian| Was this computed on a \verb|Little| or \verb|Big| Endian machine.

\verb|Precision| Is the propagator \verb|Single| or \verb|Double| precision?
Our code stores everything as double precision so if the propagator is single
precision we have to do a cast.

As of writing \verb|Lattice|, \verb|Precision|, \verb|Endian|, \verb|SrcPos|,
\verb|Basis| and \verb|Source| are required quantities and the code will exit
without these in the header.

\subsection{Gauge configuration readers}

We use the readers from my other code GLU %href{}
for the gauge fields as well as plaquette checks and what have you.
I guess this code was first introduced in \cite{Hudspith:2014oja}.

The code reads the whole lattice-wide gauge field into memory so this
can get a little expensive when reading large lattices.

It supports the following gauge field types,
\begin{itemize}
\item NERSC gauge field (\verb|NERSC_GAUGE NERSC_NCxNC|) %\href.
\item ILDG/Scidac/Lime (\verb|Scidac ILDG_SCIDAC ILDG_BQCD LIME|) %\href.
\item HiRep (\verb|HIREP|) \cite{} %\href.
\item Unit gauge field (\verb|UNIT|).
\end{itemize}

Where the quantities in parentheses are the various \verb|HEADER_MODE| settings
required in the input file.

The code attempts to check the various checksums available, apart from if
using the LIME reader (i.e. you are reading something that is not standard). 
It will tell you the read is unsafe but continue anyway.

The code outputs the dimensions read from the configuration file as well as
computing the average plaquette and average link trace.

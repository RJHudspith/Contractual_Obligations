\section{Geometry}

This section defines our default $\gamma$-matrix definitions and our lattice
geometry and finally discusses the binding to FFTW \cite{} %href{}

\subsection{Gamma definitions}

Our gamma matrices are stored as two 8-bit integer arrays each of length $NS$.
We record the non-zero column index ``i'' and an integer representing on of the
four roots of unity ``ig''. This is compact and multiplication of elements can
be performed by modular arithmetic.

We define our $\gamma$ matrices in $(x,y,z,t..)$ format as we do our gauge
field, so $\gamma_0=\gamma_x$ and so on. We also define 
$\gamma_5=\gamma_0\gamma_1\gamma_2\gamma_3$. Here is the list of our gamma
conventions.

\subsection{Chiral basis}

\begin{equation}
\begin{gathered}
%% gamma_0
\mbox{ \footnotesize$
\gamma_0=
\left[
\begin{array}{cccc}
0 & 0 & 0 & -i \\
0 & 0 & -i & 0 \\
0 & +i & 0 & 0 \\
+i & 0 & 0 & 0 \\
\end{array}\right]
$\normalsize }, \quad
%% gamma_1
\mbox{ \footnotesize$
\gamma_1=
\left[
\begin{array}{cccc}
0 & 0 & 0 & -1 \\
0 & 0 & +1 & 0 \\
0 & +1 & 0 & 0 \\
-1 & 0 & 0 & 0 \\
\end{array}\right]
$\normalsize }, \quad
%% gamma_2
\mbox{ \footnotesize$
\gamma_2=
\left[
\begin{array}{cccc}
0 & 0 & -i & 0 \\
0 & 0 & 0 & +i \\
+i & 0 & 0 & 0 \\
0 & -i & 0 & 0 \\
\end{array}\right]
$\normalsize }, \\
%% gamma_3
\mbox{ \footnotesize$
\gamma_3=
\left[
\begin{array}{cccc}
0 & 0 & -1 & 0 \\
0 & 0 & 0 & -1 \\
-1 & 0 & 0 & 0 \\
0 & -1 & 0 & 0 \\
\end{array}\right]
$\normalsize }, \quad
%% identity
\mbox{ \footnotesize$
\gamma_4=
\left[
\begin{array}{cccc}
+1 & 0 & 0 & 0 \\
0 & +1 & 0 & 0 \\
0 & 0 & +1 & 0 \\
0 & 0 & 0 & +1 \\
\end{array}\right]
$\normalsize }, \quad
\mbox{ \footnotesize$
\gamma_5=
\left[
\begin{array}{cccc}
-1 & 0 & 0 & 0 \\
0 & -1 & 0 & 0 \\
0 & 0 & +1 & 0 \\
0 & 0 & 0 & +1 \\
\end{array}\right]
$\normalsize }. \\
\end{gathered}
\end{equation}

\subsection{Non-relativistic basis}

\begin{equation}
\begin{gathered}
%% gamma_0
\mbox{ \footnotesize$
\gamma_0=
\left[
\begin{array}{cccc}
0 & 0 & 0 & -i \\
0 & 0 & -i & 0 \\
0 & +i & 0 & 0 \\
+i & 0 & 0 & 0 \\
\end{array}\right]
$\normalsize }, \quad
%% gamma_1
\mbox{ \footnotesize$
\gamma_1=
\left[
\begin{array}{cccc}
0 & 0 & 0 & -1 \\
0 & 0 & +1 & 0 \\
0 & +1 & 0 & 0 \\
-1 & 0 & 0 & 0 \\
\end{array}\right]
$\normalsize }, \quad
%% gamma_2
\mbox{ \footnotesize$
\gamma_2=
\left[
\begin{array}{cccc}
0 & 0 & -i & 0 \\
0 & 0 & 0 & +i \\
+i & 0 & 0 & 0 \\
0 & -i & 0 & 0 \\
\end{array}\right]
$\normalsize }, \\
%% gamma_3
\mbox{ \footnotesize$
\gamma_3=
\left[
\begin{array}{cccc}
+1 & 0 & 0 & 0 \\
0 & +1 & 0 & 0 \\
0 & 0 & -1 & 0 \\
0 & 0 & 0 & -1 \\
\end{array}\right]
$\normalsize }, \quad
%% identity
\mbox{ \footnotesize$
\gamma_4=
\left[
\begin{array}{cccc}
+1 & 0 & 0 & 0 \\
0 & +1 & 0 & 0 \\
0 & 0 & +1 & 0 \\
0 & 0 & 0 & +1 \\
\end{array}\right]
$\normalsize }, \quad
\mbox{ \footnotesize$
\gamma_5=
\left[
\begin{array}{cccc}
0 & 0 & -1 & 0 \\
0 & 0 & 0 & -1 \\
-1 & 0 & 0 & 0 \\
0 & -1 & 0 & 0 \\
\end{array}\right]
$\normalsize }. \\
\end{gathered}
\end{equation}

The further $NS*NS$ elements of our gamma basis are defined as products of 
these elementary gamma-matrices.

i.e. we have,
\begin{equation}
\begin{gathered}
\gamma_6 = \gamma_0 \gamma_5, 
\quad \gamma_7 = \gamma_1 \gamma_5, 
\quad \gamma_8 = \gamma_2 \gamma_5, \\
\gamma_9 = \gamma_3 \gamma_5,
\quad \gamma_{10} = \gamma_0\gamma_1, 
\quad \gamma_{11} = \gamma_0\gamma_2,
\quad \gamma_{12} = \gamma_0 \gamma_3,\\
\quad \gamma_{13} = \gamma_1 \gamma_2,
\quad \gamma_{14} = \gamma_1 \gamma_3,
\quad \gamma_{15} = \gamma_2 \gamma_3.
\end{gathered}
\end{equation}

\subsection{Geometry and lattice dimensions}

We take the geometry functions from my other library GLU. As they were designed
for navigating around gauge fields they assume periodicity in all dimensions.
This is not a problem but should be taken into account if using some weird
boundary conditions.

For a $ND=4$ gauge theory we assume dimension labelling of $(x,y,z,t)$ where
x runs fastest and t slowest in row-major ordering. We flatten our site index
to lexicographical row-major order, i.e.
\begin{equation}
\text{site} = x + L_x( y + L_y ( z + L_z t ) ).
\end{equation}
Where $L_x$ is the length of the lattice in the x-direction.

\subsection{Momenta}

We have the library FFTW do all of our Fourier-transforming needs. Although we
need to be careful as they define their first Brillouin Zone as between
$0$ and $2\pi$ whereas we want it to be between $-\pi$ and $\pi$. As the DFT
is periodic there is no difference in these two definitions, it just requires a
mapping of momenta from $\pi \rightarrow 2\pi$ to $-\pi -> 0$ functions to do
this are available in the code.

Lattice momenta are often defined in one of the variants,
\begin{equation}
p_i = \frac{2\pi n_i}{L_i},\quad \tilde{p_i}=\sin\left(\frac{2\pi n_i}{L_i}\right),
\quad \hat{p_i}=2\sin\left(\frac{\pi n_i}{L_i}\right).
\end{equation}
Depending on the quantity you are measuring. The $n's$ are the Fourier modes 
(i.e. integer $ND$-tuples), the p's are the discrete momenta.

\subsection{FFTW binding}

\subsubsection{Discrete Fourier transforms}

A 1-dimensional DFT is defined by,
\begin{equation}
f(p) = \sum_{x=0}^{L_x-1}e^{(\text{s})ipx}f(x).
\end{equation}
Where (s) is the signe of the transform, we use forward s=1 and backward s=-1.
Generalising to an $ND$-dimensional DFT is simple,
\begin{equation}
f(p) = \sum_r^{L_{ND-1}-1} ... \sum_{y=0}^{L_y-1} \sum_{x=0}^{L_x-1}e^{(s)i(p_x x + p_y y + ... p_r r)}f(x).
\end{equation}

This can be done by brute force but after one or two different momenta, it is
much more economical to use an FFT.

\subsubsection{Wrappers and wisdom}

We use wrappers for our FFTW binding, these create the plans necessary for the ffts.
FFTW works by having many different variants/kernels of FFTs coded and searching
for the fastest, this requires the code to test many FFTs over the volume specified
which can be slow for large volumes. If \verb|--enable-notcondor| is set then the
fastest FFT type is saved to a file and used the next time the code is called instead
of it running the tests. This type of file cacheing is used a bit in our code (the
momentum list after a cut is stored too) and the saved files can be found in 
\verb|$install/Local/|.

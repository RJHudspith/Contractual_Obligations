\section{Vacuum polarisation functions}

This section discusses our implementation of the various currents
we have implemented for the computation of the vacuum polarisation
function.

\subsection{Conserved non-local Wilson currents}

For Wilson fermions one can define a conserved vector current, i.e.
one that satisfies a vector Ward identity.

\subsubsection{Ward identities}

The Ward identity $\partial_\mu \mathfrak{V}_\mu = 0$
in continuum can be translated to the lattice via finite difference
$\delta_\mu \mathfrak{V}_\mu = 0 $ we can define a ``backward'' current,
\begin{equation}
2\mathfrak{V}_\mu=
\bar{\Psi}\left(x+a\hat{\mu}\right)\left( 1 + \gamma_\mu \right)U_\mu^{\dagger}\left(x+a\frac{\mu}{2}\right)\Psi(x)
-\bar{\Psi}(x)\left( 1 - \gamma_\mu \right)U_\mu\left(x+a\frac{\mu}{2}\right)\Psi\left(x+a\hat{\mu}\right).
\end{equation}
Which satisfies the backward (-) finite difference,
\begin{equation}
\delta_\mu^{(-)}\mathfrak{V}_\mu(x) = \mathfrak{V}_\mu(x) - \mathfrak{V}_\mu(x-a\hat{\mu}) = 0.
\end{equation}

One could also define the ``forward'' current which satisfies
a forward finite difference,
\begin{equation}
2\mathfrak{V}_\mu=
\bar{\Psi}(x)\left( 1 + \gamma_\mu \right)U_\mu^{\dagger}\left(x-a\frac{\mu}{2}\right)\Psi\left(x-a\hat{\mu}\right)
-\bar{\Psi}\left(x-a\hat{\mu}\right)\left( 1 - \gamma_\mu \right)U_\mu\left(x-a\frac{\mu}{2}\right)\Psi(x).
\end{equation}
With forward (+) finite difference,
\begin{equation}
\delta_\mu^{(+)}\mathfrak{V}_\mu(x) = \mathfrak{V}_\mu(x+a\hat{\mu}) - \mathfrak{V}_\mu(x) = 0.
\end{equation}

\subsection{Conserved-local currents}

A popular current to compute is a conserved-local current. This is
because it only requires 1 propagator inversion whereas a 
conserved-conserved current would require $ND+1$.

The idea is to contract one of the currents above with a local
current sat at zero,
\begin{equation}
V_\mu(0,0) = \bar{\Psi}(0,0) \gamma_\mu \Psi(0,0).
\end{equation}

We make the wick contraction of a (flavour diagonal) backward conserved-local vector current,
\begin{equation}
\begin{aligned}
2\mathfrak{V}_\mu(x,t)V_\nu(0,0) =& 
-\text{Tr}\left[ \gamma_\nu \left( \gamma_5 S(x,t) \gamma_5 \right)^{\dagger} U_\mu\left( x + a\frac{\hat{\mu}}{2}\right) S(x+a\hat{\mu},t) \right] \\
& + \text{Tr}\left[ \gamma_\nu \left( \gamma_5 S(x,t) \gamma_5 \right)^{\dagger} \gamma_\mu U_\mu\left( x + a\frac{\hat{\mu}}{2}\right) S(x+a\hat{\mu},t) \right] \\
& + \text{Tr}\left[ \gamma_\nu \left( \gamma_5 S(x+a\hat{\mu},t) \gamma_5 \right)^{\dagger} U_\mu^{\dagger}\left( x + a\frac{\hat{\mu}}{2}\right) S(x,t) \right] \\
& + \text{Tr}\left[ \gamma_\nu \left( \gamma_5 S(x+a\hat{\mu},t) \gamma_5 \right)^{\dagger} \gamma_\mu U_\mu^{\dagger}\left( x + a\frac{\hat{\mu}}{2}\right) S(x,t) \right].
\end{aligned}
\end{equation}

The benefit of writing the contraction like this is that once we have (pre)computed the products,
$U_\mu\left( x + a\frac{\hat{\mu}}{2}\right) S(x+a\hat{\mu},t)$ and
$U_\mu^{\dagger}\left( x + a\frac{\hat{\mu}}{2}\right) S(x,t)$ our contraction is exactly that of
the case of the mesons and we can use the optimised contraction routines from there.

\subsection{Non-conserved non-local Wilson currents}

We can also define non-local non-conserved Wilson currents. Of some interest is the axial,
\begin{equation}
\begin{aligned}
2\mathfrak{A}_\mu^{(\text{NCNL})}(x,t)A_\nu(0,0) =& 
+\text{Tr}\left[ \gamma_\nu \gamma_5 \left( \gamma_5 S(x,t) \gamma_5 \right)^{\dagger} \gamma_\mu \gamma_5 U_\mu\left( x + a\frac{\hat{\mu}}{2}\right) S(x+a\hat{\mu},t) \right] \\
&+ \text{Tr}\left[ \gamma_\nu \gamma_5 \left( \gamma_5 S(x+a\hat{\mu},t) \gamma_5 \right)^{\dagger} \gamma_\mu \gamma_5 U_\mu^{\dagger}\left( x + a\frac{\hat{\mu}}{2}\right) S(x,t) \right].
\end{aligned}
\end{equation}
Which is not conserved.

Also one can write down the non-local non-conserved vector,
\begin{equation}
\begin{aligned}
2\mathfrak{V}_\mu^{(\text{NCNL})}(x,t)V_\nu(0,0) =& 
+\text{Tr}\left[ \gamma_\nu \left( \gamma_5 S(x,t) \gamma_5 \right)^{\dagger} \gamma_\mu U_\mu\left( x + a\frac{\hat{\mu}}{2}\right) S(x+a\hat{\mu},t) \right] \\
&+ \text{Tr}\left[ \gamma_\nu \left( \gamma_5 S(x+a\hat{\mu},t) \gamma_5 \right)^{\dagger} \gamma_\mu U_\mu^{\dagger}\left( x + a\frac{\hat{\mu}}{2}\right) S(x,t) \right].
\end{aligned}
\end{equation}

\subsection{Temporal correlators}

Some people like to work with the zero spatial momentum projected current correlators,
\begin{equation}
\Pi_{\mu\nu}(0,t) = \frac{1}{2}\sum_{x}^{V_{ND-1}} \mathfrak{V}_\mu(x,t)V_\nu(0,0).
\end{equation}

This can then be used to compute the moments of the correlator. Our code computes this
by default as well as full volume-wide FFTs.

\subsection{Momentum-space currents}

We also produce the full $ND$-volume FFT of our current-correlator, although this is
only useful for point sources as extended sources such as the Walls explicitly zero-
momentum project, as mentioned in the mesons section.

We define the momentum-space correlator as,
\begin{equation}
\Pi_{\mu\nu}(q) = \frac{1}{2}e^{-iaq_\mu/2}\sum_{x}^{V}e^{iaq_\mu \cdot x} \mathfrak{V}_\mu(x)V_\nu(0).
\end{equation}
Note how we no longer distinguish between the spatial and temporal components.

The factor $e^{-iaq_\mu/2}$ is to ensure the correct momentum-space Ward identity.
Considering the two Fourier transforms,
\begin{equation}
\begin{aligned}
&\sum_{x}^{V}e^{iaq\cdot x}\mathfrak{V}_{\mu}(x) = \mathfrak{V}_{\mu}(q), \\
&\sum_{x}^{V}e^{iaq\cdot x}\mathfrak{V}_{\mu}(x-a\hat{\mu}) = e^{-iaq_\mu}\mathfrak{V}_{\mu}(q).
\end{aligned}
\end{equation}
Taking the finite difference and Fourier transforming we have,
\begin{equation}
\begin{aligned}
\sum_{\mu}\sum_{x}^{V}e^{iaq\cdot x}\left( \mathfrak{V}_{\mu}(x)- \mathfrak{V}_{\mu}(x-a\hat{\mu}) \right) &= \sum_\mu \left( \mathfrak{V}_{\mu}(q) - e^{-iaq_\mu}\mathfrak{V}_{\mu}(q) \right), \\
&= \sum_\mu e^{-iaq_\mu/2}\left( e^{iaq_\mu /2}\mathfrak{V}_{\mu}(q) - e^{-iaq_\mu /2}\mathfrak{V}_{\mu}(q) \right),\\
&= \sum_\mu e^{-iaq_\mu/2} 2i \sin( aq_\mu / 2 )\mathfrak{V}_{\mu}(q) = 0.\\
\end{aligned}
\end{equation}
In comparison to the continuum ward identity $q_\mu \mathfrak{V}_\mu(q)$ we have a natural
expression for the momentum of our current, that of $a\hat{q}_\mu = 2 \sin( aq_\mu / 2 )$ as
this is the only momentum definition that satisfies the Ward identity.

\subsubsection{Projections}

We use the method of projection to compute various scalar quantities from our currents.
In continuum, the Lorentz structure of the vacuum polarisation function is
\begin{equation}
\Pi_{\mu\nu}(q) = \left( q^2 \delta_{\mu\nu} - q_\mu q_\nu \right)\Pi^{(1)}(q^2) - q_\mu q_\nu \Pi^{(0)}(q^2).
\end{equation}
Where $\Pi^{(1)}(q^2)$ is the Transverse component of the vacuum polarisation function
and $\Pi^{(0)}(q^2)$ is the Longitudinal. The conserved-local vector current is purely
transverse as it satisfies a Ward identity, but other non-conserved currents can have
longitudinal components.

It seems natural to write the structure in terms of our lattice momentum,
\begin{equation}
\Pi_{\mu\nu}(a\hat{q}) = \left( \left(a\hat{q}\right)^2 \delta_{\mu\nu} - a^2\hat{q}_\mu \hat{q}_\nu \right)\Pi^{(1)}(a^2\hat{q}^2) - a^2\hat{q}_\mu \hat{q}_\nu \Pi^{(0)}(a^2\hat{q}^2) + \text{h.o.t}.
\end{equation}
Where I have implicitly allowed for possible higher order terms (h.o.t) which come from
the violation of lorentz symmetry.

We define the following projections to extract the longitudinal and transverse scalar
form factors,
\begin{equation}
\begin{aligned}
\Pi^{(0)}(a^2\hat{q}^2) &= -\frac{ a^2\hat{q}_\mu \hat{q}_\nu }{(a^2\hat{q}^2)^2} \Pi_{\mu\nu}(aq), \\
\Pi^{(1)}(a^2\hat{q}^2) &= \frac{1}{(ND-1)a^2\hat{q}^2}\left( \delta_{\mu\nu} - \frac{ \hat{q}_\mu \hat{q}_\nu}{\hat{q}^2}\right)\Pi_{\mu\nu}(aq).
\end{aligned}
\end{equation}
